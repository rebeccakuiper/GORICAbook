% Options for packages loaded elsewhere
\PassOptionsToPackage{unicode}{hyperref}
\PassOptionsToPackage{hyphens}{url}
\documentclass[
  14pt,
]{article}
\usepackage{xcolor}
\usepackage[margin=1in]{geometry}
\usepackage{amsmath,amssymb}
\setcounter{secnumdepth}{5}
\usepackage{iftex}
\ifPDFTeX
  \usepackage[T1]{fontenc}
  \usepackage[utf8]{inputenc}
  \usepackage{textcomp} % provide euro and other symbols
\else % if luatex or xetex
  \usepackage{unicode-math} % this also loads fontspec
  \defaultfontfeatures{Scale=MatchLowercase}
  \defaultfontfeatures[\rmfamily]{Ligatures=TeX,Scale=1}
\fi
\usepackage{lmodern}
\ifPDFTeX\else
  % xetex/luatex font selection
\fi
% Use upquote if available, for straight quotes in verbatim environments
\IfFileExists{upquote.sty}{\usepackage{upquote}}{}
\IfFileExists{microtype.sty}{% use microtype if available
  \usepackage[]{microtype}
  \UseMicrotypeSet[protrusion]{basicmath} % disable protrusion for tt fonts
}{}
\makeatletter
\@ifundefined{KOMAClassName}{% if non-KOMA class
  \IfFileExists{parskip.sty}{%
    \usepackage{parskip}
  }{% else
    \setlength{\parindent}{0pt}
    \setlength{\parskip}{6pt plus 2pt minus 1pt}}
}{% if KOMA class
  \KOMAoptions{parskip=half}}
\makeatother
\usepackage{color}
\usepackage{fancyvrb}
\newcommand{\VerbBar}{|}
\newcommand{\VERB}{\Verb[commandchars=\\\{\}]}
\DefineVerbatimEnvironment{Highlighting}{Verbatim}{commandchars=\\\{\}}
% Add ',fontsize=\small' for more characters per line
\usepackage{framed}
\definecolor{shadecolor}{RGB}{248,248,248}
\newenvironment{Shaded}{\begin{snugshade}}{\end{snugshade}}
\newcommand{\AlertTok}[1]{\textcolor[rgb]{0.94,0.16,0.16}{#1}}
\newcommand{\AnnotationTok}[1]{\textcolor[rgb]{0.56,0.35,0.01}{\textbf{\textit{#1}}}}
\newcommand{\AttributeTok}[1]{\textcolor[rgb]{0.13,0.29,0.53}{#1}}
\newcommand{\BaseNTok}[1]{\textcolor[rgb]{0.00,0.00,0.81}{#1}}
\newcommand{\BuiltInTok}[1]{#1}
\newcommand{\CharTok}[1]{\textcolor[rgb]{0.31,0.60,0.02}{#1}}
\newcommand{\CommentTok}[1]{\textcolor[rgb]{0.56,0.35,0.01}{\textit{#1}}}
\newcommand{\CommentVarTok}[1]{\textcolor[rgb]{0.56,0.35,0.01}{\textbf{\textit{#1}}}}
\newcommand{\ConstantTok}[1]{\textcolor[rgb]{0.56,0.35,0.01}{#1}}
\newcommand{\ControlFlowTok}[1]{\textcolor[rgb]{0.13,0.29,0.53}{\textbf{#1}}}
\newcommand{\DataTypeTok}[1]{\textcolor[rgb]{0.13,0.29,0.53}{#1}}
\newcommand{\DecValTok}[1]{\textcolor[rgb]{0.00,0.00,0.81}{#1}}
\newcommand{\DocumentationTok}[1]{\textcolor[rgb]{0.56,0.35,0.01}{\textbf{\textit{#1}}}}
\newcommand{\ErrorTok}[1]{\textcolor[rgb]{0.64,0.00,0.00}{\textbf{#1}}}
\newcommand{\ExtensionTok}[1]{#1}
\newcommand{\FloatTok}[1]{\textcolor[rgb]{0.00,0.00,0.81}{#1}}
\newcommand{\FunctionTok}[1]{\textcolor[rgb]{0.13,0.29,0.53}{\textbf{#1}}}
\newcommand{\ImportTok}[1]{#1}
\newcommand{\InformationTok}[1]{\textcolor[rgb]{0.56,0.35,0.01}{\textbf{\textit{#1}}}}
\newcommand{\KeywordTok}[1]{\textcolor[rgb]{0.13,0.29,0.53}{\textbf{#1}}}
\newcommand{\NormalTok}[1]{#1}
\newcommand{\OperatorTok}[1]{\textcolor[rgb]{0.81,0.36,0.00}{\textbf{#1}}}
\newcommand{\OtherTok}[1]{\textcolor[rgb]{0.56,0.35,0.01}{#1}}
\newcommand{\PreprocessorTok}[1]{\textcolor[rgb]{0.56,0.35,0.01}{\textit{#1}}}
\newcommand{\RegionMarkerTok}[1]{#1}
\newcommand{\SpecialCharTok}[1]{\textcolor[rgb]{0.81,0.36,0.00}{\textbf{#1}}}
\newcommand{\SpecialStringTok}[1]{\textcolor[rgb]{0.31,0.60,0.02}{#1}}
\newcommand{\StringTok}[1]{\textcolor[rgb]{0.31,0.60,0.02}{#1}}
\newcommand{\VariableTok}[1]{\textcolor[rgb]{0.00,0.00,0.00}{#1}}
\newcommand{\VerbatimStringTok}[1]{\textcolor[rgb]{0.31,0.60,0.02}{#1}}
\newcommand{\WarningTok}[1]{\textcolor[rgb]{0.56,0.35,0.01}{\textbf{\textit{#1}}}}
\usepackage{longtable,booktabs,array}
\usepackage{calc} % for calculating minipage widths
% Correct order of tables after \paragraph or \subparagraph
\usepackage{etoolbox}
\makeatletter
\patchcmd\longtable{\par}{\if@noskipsec\mbox{}\fi\par}{}{}
\makeatother
% Allow footnotes in longtable head/foot
\IfFileExists{footnotehyper.sty}{\usepackage{footnotehyper}}{\usepackage{footnote}}
\makesavenoteenv{longtable}
\usepackage{graphicx}
\makeatletter
\newsavebox\pandoc@box
\newcommand*\pandocbounded[1]{% scales image to fit in text height/width
  \sbox\pandoc@box{#1}%
  \Gscale@div\@tempa{\textheight}{\dimexpr\ht\pandoc@box+\dp\pandoc@box\relax}%
  \Gscale@div\@tempb{\linewidth}{\wd\pandoc@box}%
  \ifdim\@tempb\p@<\@tempa\p@\let\@tempa\@tempb\fi% select the smaller of both
  \ifdim\@tempa\p@<\p@\scalebox{\@tempa}{\usebox\pandoc@box}%
  \else\usebox{\pandoc@box}%
  \fi%
}
% Set default figure placement to htbp
\def\fps@figure{htbp}
\makeatother
\setlength{\emergencystretch}{3em} % prevent overfull lines
\providecommand{\tightlist}{%
  \setlength{\itemsep}{0pt}\setlength{\parskip}{0pt}}
\usepackage[]{natbib}
\bibliographystyle{plainnat}
\usepackage{booktabs}
\usepackage{amsthm}
\makeatletter
\def\thm@space@setup{%
  \thm@preskip=8pt plus 2pt minus 4pt
  \thm@postskip=\thm@preskip
}
\makeatother
\usepackage{bookmark}
\IfFileExists{xurl.sty}{\usepackage{xurl}}{} % add URL line breaks if available
\urlstyle{same}
\hypersetup{
  pdftitle={Guidelines for GORIC(A) input},
  pdfauthor={Rebecca M. Kuiper and Leonard Vanbrabant},
  hidelinks,
  pdfcreator={LaTeX via pandoc}}

\title{Guidelines for GORIC(A) input}
\author{Rebecca M. Kuiper and Leonard Vanbrabant}
\date{19 februari 2026}

\begin{document}
\maketitle

{
\setcounter{tocdepth}{2}
\tableofcontents
}
\section{Types of input}\label{types-of-input}

The \texttt{goric} function of \texttt{restriktor} takes different forms of input:\\
* enter fitted unconstrained object:\\
- lm,\\
- rlm,\\
- glm,\\
- glmer(Mod),\\
- nlmer(Mod),\\
- lmer(Mod),\\
- CTmeta,\\
- rma,\\
- lavaan;
* enter the (structural) parameter estimates (possibly, standardized) and their covariance matrix.\\
Functions that can often be used for extrcation:\\
- coef() \& vcov()\\
- fixef() \& vcov()

Please note that:\\
* The GORIC can only be calculated for lm objects.\\
* The GORIC (type = `goric') is the default for lm objects, while GORICA (type = `gorica') is for the other input options.\\
* In the GORICA, the estimates are assumed to be normally distributed. So, for some models when sample size is low, this assumption may not hold. In that case, it is often not clear how well the GORICA performs. See Altinisik et al 2021, for some simulations regarding logistic regression and SEM models (for which the GORICA performs well).

Next, I will generate (regression) data.
Afterwards, I will show some example for how one can extract the input needed and evaluate the hypothesis of interest based on that.

\section{Generate fictional data}\label{generate-fictional-data}

\begin{Shaded}
\begin{Highlighting}[]
\CommentTok{\# Population values regression coefficients (for 3 predictors)}
\NormalTok{coeffs }\OtherTok{\textless{}{-}} \FunctionTok{c}\NormalTok{(}\DecValTok{2}\NormalTok{, }\FloatTok{1.8}\NormalTok{, }\FloatTok{1.6}\NormalTok{)}

\CommentTok{\# Sample size}
\NormalTok{n }\OtherTok{\textless{}{-}} \DecValTok{600}

\CommentTok{\# Generate X (predictors)}
\FunctionTok{set.seed}\NormalTok{(}\DecValTok{123}\NormalTok{)}
\NormalTok{x1 }\OtherTok{\textless{}{-}} \FunctionTok{rnorm}\NormalTok{(n, }\DecValTok{0}\NormalTok{, }\DecValTok{1}\NormalTok{)}
\NormalTok{x2 }\OtherTok{\textless{}{-}} \FunctionTok{rnorm}\NormalTok{(n, }\DecValTok{0}\NormalTok{, }\DecValTok{1}\NormalTok{)}
\NormalTok{x3 }\OtherTok{\textless{}{-}} \FunctionTok{rnorm}\NormalTok{(n, }\DecValTok{0}\NormalTok{, }\DecValTok{1}\NormalTok{)}
\NormalTok{data\_unstand }\OtherTok{\textless{}{-}} \FunctionTok{cbind}\NormalTok{(x1, x2, x3)}
\CommentTok{\# Standardize data {-} since parameters for continuous variables will be compared}
\NormalTok{data }\OtherTok{\textless{}{-}} \FunctionTok{as.data.frame}\NormalTok{(}\FunctionTok{scale}\NormalTok{(data\_unstand))  }\CommentTok{\# Standardized!}

\CommentTok{\# Generate y (outcome)}
\NormalTok{y }\OtherTok{\textless{}{-}}\NormalTok{ coeffs[}\DecValTok{1}\NormalTok{] }\SpecialCharTok{*}\NormalTok{ data}\SpecialCharTok{$}\NormalTok{x1 }\SpecialCharTok{+}\NormalTok{ coeffs[}\DecValTok{2}\NormalTok{] }\SpecialCharTok{*}\NormalTok{ data}\SpecialCharTok{$}\NormalTok{x2 }\SpecialCharTok{+}\NormalTok{ coeffs[}\DecValTok{3}\NormalTok{] }\SpecialCharTok{*}\NormalTok{ data}\SpecialCharTok{$}\NormalTok{x3 }\SpecialCharTok{+} \FunctionTok{rnorm}\NormalTok{(n)}
\NormalTok{data}\SpecialCharTok{$}\NormalTok{y }\OtherTok{\textless{}{-}}\NormalTok{ y}

\CommentTok{\# To also use lmer / lme4, we also need:}
\NormalTok{data\_ID }\OtherTok{\textless{}{-}}\NormalTok{ data}
\NormalTok{data\_ID}\SpecialCharTok{$}\NormalTok{ID }\OtherTok{\textless{}{-}} \DecValTok{1}\SpecialCharTok{:}\NormalTok{(}\FunctionTok{dim}\NormalTok{(data)[}\DecValTok{1}\NormalTok{]}\SpecialCharTok{/}\DecValTok{10}\NormalTok{)}
\end{Highlighting}
\end{Shaded}

\section{Obtain input \& output for GORICA}\label{obtain-input-output-for-gorica}

Next, I will show how one can extract the input needed and evaluate the hypothesis of interest based on that.

\begin{Shaded}
\begin{Highlighting}[]
\CommentTok{\# Hypothesis of interest using the default R labeling when predictors are}
\CommentTok{\# called x1, x2, and x3}
\NormalTok{H1 }\OtherTok{\textless{}{-}} \StringTok{"x1 \textgreater{} x2 \textgreater{} x3"}
\end{Highlighting}
\end{Shaded}

\subsection{lm}\label{lm}

\begin{Shaded}
\begin{Highlighting}[]
\NormalTok{fit.lm }\OtherTok{\textless{}{-}} \FunctionTok{lm}\NormalTok{(y }\SpecialCharTok{\textasciitilde{}} \DecValTok{1} \SpecialCharTok{+}\NormalTok{ x1 }\SpecialCharTok{+}\NormalTok{ x2 }\SpecialCharTok{+}\NormalTok{ x3, }\AttributeTok{data =}\NormalTok{ data)}
\end{Highlighting}
\end{Shaded}

\subsubsection{fit object}\label{fit-object}

\begin{Shaded}
\begin{Highlighting}[]
\CommentTok{\# Apply GORIC \#}
\FunctionTok{set.seed}\NormalTok{(}\DecValTok{123}\NormalTok{)}
\NormalTok{goric\_lm }\OtherTok{\textless{}{-}} \FunctionTok{goric}\NormalTok{(fit.lm, }\AttributeTok{hypotheses =} \FunctionTok{list}\NormalTok{(H1))}
\NormalTok{goric\_lm}
\end{Highlighting}
\end{Shaded}

\begin{verbatim}
restriktor (0.6-30): generalized order-restricted information criterion: 

Results:
        model    loglik  penalty     goric  loglik.weights  penalty.weights  goric.weights
1          H1  -834.834    3.815  1677.297           0.989            0.705          0.995
2  complement  -839.346    4.685  1688.063           0.011            0.295          0.005

Conclusion:
The order-restricted hypothesis 'H1' has 217.69 times more support than its complement.
\end{verbatim}

\begin{Shaded}
\begin{Highlighting}[]
\CommentTok{\# Apply GORICA \# When using an lm object, then by default goric; so overrule:}
\FunctionTok{set.seed}\NormalTok{(}\DecValTok{123}\NormalTok{)}
\NormalTok{gorica\_lm }\OtherTok{\textless{}{-}} \FunctionTok{goric}\NormalTok{(fit.lm, }\AttributeTok{hypotheses =} \FunctionTok{list}\NormalTok{(H1), }\AttributeTok{type =} \StringTok{"gorica"}\NormalTok{)}
\NormalTok{gorica\_lm}
\end{Highlighting}
\end{Shaded}

\begin{verbatim}
restriktor (0.6-30): generalized order-restricted information criterion approximation:

Results:
        model  loglik  penalty   gorica  loglik.weights  penalty.weights  gorica.weights
1          H1   9.222    2.815  -12.815           0.990            0.705           0.996
2  complement   4.676    3.685   -1.980           0.010            0.295           0.004

Conclusion:
The order-restricted hypothesis 'H1' has 225.25 times more support than its complement.
\end{verbatim}

\begin{Shaded}
\begin{Highlighting}[]
\CommentTok{\# Note that this is currently based on a covariance matrix using N{-}k instead of}
\CommentTok{\# N (see also below).  TO DO}
\end{Highlighting}
\end{Shaded}

\subsubsection{extract estimates}\label{extract-estimates}

\begin{Shaded}
\begin{Highlighting}[]
\NormalTok{est\_lm }\OtherTok{\textless{}{-}} \FunctionTok{coef}\NormalTok{(fit.lm)}
\NormalTok{VCOV\_lm }\OtherTok{\textless{}{-}} \FunctionTok{vcov}\NormalTok{(fit.lm)}
\CommentTok{\# Since lm and lmer use N{-}k instead of N, with k the number of (regression)}
\CommentTok{\# coefficients to be estimated:}
\NormalTok{N }\OtherTok{\textless{}{-}} \FunctionTok{dim}\NormalTok{(data)[}\DecValTok{1}\NormalTok{]  }\CommentTok{\# = n = 600}
\NormalTok{k }\OtherTok{\textless{}{-}} \FunctionTok{dim}\NormalTok{(data)[}\DecValTok{2}\NormalTok{]  }\CommentTok{\# = 1+3, namely: 1 intercept + 3 regression slopes}
\NormalTok{VCOV\_lm }\OtherTok{\textless{}{-}} \FunctionTok{vcov}\NormalTok{(fit.lm) }\SpecialCharTok{*}\NormalTok{ (N }\SpecialCharTok{{-}}\NormalTok{ k)}\SpecialCharTok{/}\NormalTok{N}
\CommentTok{\# VCOV\_lm \textless{}{-} vcov(fit.lm)*(N{-}1)/N TO DO}

\CommentTok{\# Apply GORICA \#}
\FunctionTok{set.seed}\NormalTok{(}\DecValTok{123}\NormalTok{)}
\NormalTok{gorica\_lm\_est }\OtherTok{\textless{}{-}} \FunctionTok{goric}\NormalTok{(est\_lm, }\AttributeTok{VCOV =}\NormalTok{ VCOV\_lm, }\AttributeTok{hypotheses =} \FunctionTok{list}\NormalTok{(H1))}
\NormalTok{gorica\_lm\_est}
\end{Highlighting}
\end{Shaded}

\begin{verbatim}
restriktor (0.6-30): generalized order-restricted information criterion approximation:

Results:
        model  loglik  penalty   gorica  loglik.weights  penalty.weights  gorica.weights
1          H1   9.222    2.815  -12.815           0.990            0.705           0.996
2  complement   4.676    3.685   -1.980           0.010            0.295           0.004

Conclusion:
The order-restricted hypothesis 'H1' has 225.25 times more support than its complement.
\end{verbatim}

\subsection{lavaan}\label{lavaan}

\begin{Shaded}
\begin{Highlighting}[]
\FunctionTok{library}\NormalTok{(lavaan)}
\end{Highlighting}
\end{Shaded}

\begin{verbatim}
This is lavaan 0.6-21
lavaan is FREE software! Please report any bugs.
\end{verbatim}

\begin{Shaded}
\begin{Highlighting}[]
\NormalTok{fit.sem }\OtherTok{\textless{}{-}} \FunctionTok{sem}\NormalTok{(}\StringTok{"y \textasciitilde{} 1 + x1 + x2 + x3"}\NormalTok{, }\AttributeTok{data =}\NormalTok{ data)}

\CommentTok{\# The default labeling of lavaan / sem() is:}
\FunctionTok{names}\NormalTok{(}\FunctionTok{coef}\NormalTok{(fit.sem))}
\end{Highlighting}
\end{Shaded}

\begin{verbatim}
[1] "y~1"  "y~x1" "y~x2" "y~x3" "y~~y"
\end{verbatim}

\begin{Shaded}
\begin{Highlighting}[]
\CommentTok{\# The \textasciitilde{} can (currently) not be used in the hypothesis, therefore, you want to}
\CommentTok{\# label the parameters (of interest) yourself:}
\NormalTok{fit.sem }\OtherTok{\textless{}{-}} \FunctionTok{sem}\NormalTok{(}\StringTok{"y \textasciitilde{} 1 + beta1*x1 + beta2*x2 + beta3*x3"}\NormalTok{, }\AttributeTok{data =}\NormalTok{ data)}

\CommentTok{\# Now, specify the hypothesis using this labeling:}
\NormalTok{H1\_sem }\OtherTok{\textless{}{-}} \StringTok{"beta1 \textgreater{} beta2 \textgreater{} beta3"}
\end{Highlighting}
\end{Shaded}

\subsubsection{fit object}\label{fit-object-1}

\begin{Shaded}
\begin{Highlighting}[]
\CommentTok{\# Apply GORICA \#}
\FunctionTok{set.seed}\NormalTok{(}\DecValTok{123}\NormalTok{)}
\NormalTok{gorica\_sem }\OtherTok{\textless{}{-}} \FunctionTok{goric}\NormalTok{(fit.sem, }\AttributeTok{hypotheses =} \FunctionTok{list}\NormalTok{(}\AttributeTok{H1 =}\NormalTok{ H1\_sem))}
\end{Highlighting}
\end{Shaded}

\begin{verbatim}

restriktor Message: The covariance matrix of the estimates was obtained via 'vcov()'. This is the biased (restricted) sample covariance matrix, not the unbiased version based on the total sample size ('N').
\end{verbatim}

\begin{Shaded}
\begin{Highlighting}[]
\NormalTok{gorica\_sem}
\end{Highlighting}
\end{Shaded}

\begin{verbatim}
restriktor (0.6-30): generalized order-restricted information criterion approximation:

Results:
        model  loglik  penalty   gorica  loglik.weights  penalty.weights  gorica.weights
1          H1   6.915    1.815  -10.201           0.990            0.705           0.996
2  complement   2.368    2.685    0.634           0.010            0.295           0.004

Conclusion:
The order-restricted hypothesis 'H1' has 225.25 times more support than its complement.
\end{verbatim}

\subsubsection{extract estimates}\label{extract-estimates-1}

\begin{Shaded}
\begin{Highlighting}[]
\NormalTok{est\_sem }\OtherTok{\textless{}{-}} \FunctionTok{coef}\NormalTok{(fit.sem)}
\NormalTok{VCOV\_sem }\OtherTok{\textless{}{-}} \FunctionTok{vcov}\NormalTok{(fit.sem)}

\CommentTok{\# Apply GORICA \#}
\FunctionTok{set.seed}\NormalTok{(}\DecValTok{123}\NormalTok{)}
\NormalTok{gorica\_sem\_est }\OtherTok{\textless{}{-}} \FunctionTok{goric}\NormalTok{(est\_sem, }\AttributeTok{VCOV =}\NormalTok{ VCOV\_sem, }\AttributeTok{hypotheses =} \FunctionTok{list}\NormalTok{(}\AttributeTok{H1 =}\NormalTok{ H1\_sem))}
\NormalTok{gorica\_sem\_est}
\end{Highlighting}
\end{Shaded}

\begin{verbatim}
restriktor (0.6-30): generalized order-restricted information criterion approximation:

Results:
        model  loglik  penalty   gorica  loglik.weights  penalty.weights  gorica.weights
1          H1  11.210    3.815  -14.791           0.990            0.705           0.996
2  complement   6.664    4.685   -3.957           0.010            0.295           0.004

Conclusion:
The order-restricted hypothesis 'H1' has 225.25 times more support than its complement.
\end{verbatim}

\subsection{lmer / lme4}\label{lmer-lme4}

\begin{Shaded}
\begin{Highlighting}[]
\FunctionTok{library}\NormalTok{(lme4)}
\end{Highlighting}
\end{Shaded}

\begin{verbatim}
Loading required package: Matrix
\end{verbatim}

\begin{Shaded}
\begin{Highlighting}[]
\NormalTok{fit.lmer }\OtherTok{\textless{}{-}} \FunctionTok{lmer}\NormalTok{(}\StringTok{"y \textasciitilde{} 1 + x1 + x2 + x3 + (1 | ID)"}\NormalTok{, }\AttributeTok{data =}\NormalTok{ data\_ID)}
\end{Highlighting}
\end{Shaded}

\begin{verbatim}
boundary (singular) fit: see help('isSingular')
\end{verbatim}

\subsubsection{fit object}\label{fit-object-2}

\begin{Shaded}
\begin{Highlighting}[]
\CommentTok{\# Apply GORICA \#}
\FunctionTok{set.seed}\NormalTok{(}\DecValTok{123}\NormalTok{)}
\NormalTok{gorica\_lmer }\OtherTok{\textless{}{-}} \FunctionTok{goric}\NormalTok{(fit.lmer, }\AttributeTok{hypotheses =} \FunctionTok{list}\NormalTok{(H1))}
\end{Highlighting}
\end{Shaded}

\begin{verbatim}

restriktor Message: The covariance matrix of the estimates was obtained via 'vcov()'. This is the biased (restricted) sample covariance matrix, not the unbiased version based on the total sample size ('N').
\end{verbatim}

\begin{Shaded}
\begin{Highlighting}[]
\NormalTok{gorica\_lmer}
\end{Highlighting}
\end{Shaded}

\begin{verbatim}
restriktor (0.6-30): generalized order-restricted information criterion approximation:

Results:
        model  loglik  penalty   gorica  loglik.weights  penalty.weights  gorica.weights
1          H1   9.209    2.815  -12.788           0.989            0.705           0.995
2  complement   4.692    3.685   -2.014           0.011            0.295           0.005

Conclusion:
The order-restricted hypothesis 'H1' has 218.52 times more support than its complement.
\end{verbatim}

\begin{Shaded}
\begin{Highlighting}[]
\CommentTok{\# Note that this is currently based on a covariance matrix using N{-}k instead of}
\CommentTok{\# N (see also below).}
\end{Highlighting}
\end{Shaded}

\subsubsection{extract estimates}\label{extract-estimates-2}

\begin{Shaded}
\begin{Highlighting}[]
\NormalTok{est\_lmer }\OtherTok{\textless{}{-}} \FunctionTok{fixef}\NormalTok{(fit.lmer)}
\NormalTok{VCOV\_lmer }\OtherTok{\textless{}{-}} \FunctionTok{vcov}\NormalTok{(fit.lmer)}
\CommentTok{\# Since lm and lmer use N{-}k instead of N, with k the number of (regression)}
\CommentTok{\# coefficients to be estimated:}
\NormalTok{N }\OtherTok{\textless{}{-}} \FunctionTok{dim}\NormalTok{(data\_ID)[}\DecValTok{1}\NormalTok{]  }\CommentTok{\# = n = 600}
\CommentTok{\# k \textless{}{-} dim(data\_ID)[2]{-}1 \# without ID, only fixed effects, so 1+3, namely: 1}
\CommentTok{\# intercept + 3 regression slopes; or use:}
\NormalTok{k }\OtherTok{\textless{}{-}} \FunctionTok{dim}\NormalTok{(VCOV\_lmer)[}\DecValTok{2}\NormalTok{]}
\NormalTok{VCOV\_lmer }\OtherTok{\textless{}{-}} \FunctionTok{vcov}\NormalTok{(fit.lmer) }\SpecialCharTok{*}\NormalTok{ (N }\SpecialCharTok{{-}}\NormalTok{ k)}\SpecialCharTok{/}\NormalTok{N}

\CommentTok{\# Apply GORICA \#}
\FunctionTok{set.seed}\NormalTok{(}\DecValTok{123}\NormalTok{)}
\NormalTok{gorica\_lmer\_est }\OtherTok{\textless{}{-}} \FunctionTok{goric}\NormalTok{(est\_lmer, }\AttributeTok{VCOV =}\NormalTok{ VCOV\_lmer, }\AttributeTok{hypotheses =} \FunctionTok{list}\NormalTok{(H1))}
\NormalTok{gorica\_lmer\_est}
\end{Highlighting}
\end{Shaded}

\begin{verbatim}
restriktor (0.6-30): generalized order-restricted information criterion approximation:

Results:
        model  loglik  penalty   gorica  loglik.weights  penalty.weights  gorica.weights
1          H1   9.222    2.815  -12.815           0.990            0.705           0.996
2  complement   4.676    3.685   -1.980           0.010            0.295           0.004

Conclusion:
The order-restricted hypothesis 'H1' has 225.25 times more support than its complement.
\end{verbatim}

\end{document}
